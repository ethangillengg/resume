%-------------------------
% Resume in Latex
% Author : Jake Gutierrez
% Based off of: https://github.com/sb2nov/resume
% License : MIT
%------------------------
\documentclass[letterpaper,10pt]{article}

\usepackage{latexsym}
\usepackage[empty]{fullpage}
\usepackage{titlesec}
\usepackage{marvosym}
\usepackage[usenames,dvipsnames]{color}
\usepackage{verbatim}
\usepackage{enumitem}
\usepackage{fancyhdr}
\usepackage[english]{babel}
\usepackage{tabularx}
\usepackage{fontawesome5}
\usepackage{multicol}
\usepackage{fontspec}

\usepackage{tikz} % For \foreach loop
\usepackage{xcolor} % For \textcolor
\usepackage{accsupp} % For BeginAccSupp and EndAccSupp
\usepackage{graphicx} % For \clipbox

%% Base-16 color theme (https://tinted-theming.github.io/base16-gallery/)
\definecolor{base00}{HTML}{202020}
\definecolor{base01}{HTML}{2a2827}
\definecolor{base02}{HTML}{504945}
\definecolor{base03}{HTML}{5a524c}
\definecolor{base04}{HTML}{bdae93}
\definecolor{base05}{HTML}{ddc7a1}
\definecolor{base06}{HTML}{ebdbb2}
\definecolor{base07}{HTML}{fbf1c7}
\definecolor{base08}{HTML}{ea6962}
\definecolor{base09}{HTML}{e78a4e}
\definecolor{base0A}{HTML}{d8a657}
\definecolor{base0B}{HTML}{a9b665}
\definecolor{base0C}{HTML}{89b482}
\definecolor{base0D}{HTML}{7daea3}
\definecolor{base0E}{HTML}{d3869b}
\definecolor{base0F}{HTML}{bd6f3e}


% Define fonts & colors
\setmainfont{JetBrainsMono Nerd Font}
\pagecolor{base00} % background
\color{base05}     % text/bullets
% change bold color
\let\oldtextbf\textbf
\renewcommand{\textbf}[1]{\textcolor{base06}{\oldtextbf{#1}}}
% change italic color
\let\oldtextit\textit
\renewcommand{\textit}[1]{\textcolor{base06}{\oldtextit{#1}}}
\usepackage[colorlinks=true,  urlcolor=base0D]{hyperref}

% Define symbol marker (e.g. using fontawesome package)
\newcommand{\cvRatingMarker}{\faStar} % Replace this with your symbol
\setlength{\multicolsep}{-3.0pt}
\setlength{\columnsep}{-1pt}
% \input{glyphtounicode}


\pagestyle{fancy}
\fancyhf{} % clear all header and footer fields
\fancyfoot{}
\renewcommand{\headrulewidth}{0pt}
\renewcommand{\footrulewidth}{0pt}

% Adjust margins
\addtolength{\oddsidemargin}{-0.6in}
\addtolength{\evensidemargin}{-0.5in}
\addtolength{\textwidth}{1.19in}
\addtolength{\topmargin}{-.7in}
\addtolength{\textheight}{1.4in}

\urlstyle{same}

\raggedbottom
\raggedright
\setlength{\tabcolsep}{0in}

% Sections formatting
\titleformat{\section}{
	\vspace{-4pt}\scshape\raggedright\large\bfseries
}{}{0em}{}[\color{base03}\titlerule \vspace{-5pt}]

% Ensure that generate pdf is machine readable/ATS parsable
% \pdfgentounicode=1

%-------------------------
% Custom commands
\newcommand{\resumeItem}[1]{
	\item\small{
		{#1 \vspace{-2pt}}
	}
}

\newcommand{\classesList}[4]{
	\item\small{
		{#1 #2 #3 #4 \vspace{-2pt}}
	}
}

\newcommand{\resumeSubheading}[4]{
	\vspace{-2pt}\item
	\begin{tabular*}{1.0\textwidth}[t]{l@{\extracolsep{\fill}}r}
		\textbf{#1} & \textbf{\small #2} \\
		\textit{\small#3} & \textit{\small #4} \\
	\end{tabular*}\vspace{-7pt}
}

\newcommand{\resumeSubSubheading}[2]{
	\item
	\begin{tabular*}{0.97\textwidth}{l@{\extracolsep{\fill}}r}
		\textit{\small#1} & \textit{\small #2} \\
	\end{tabular*}\vspace{-7pt}
}

\newcommand{\resumeProjectHeading}[2]{
	\item
	\begin{tabular*}{1.001\textwidth}{l@{\extracolsep{\fill}}r}
		\small#1 & \textbf{\small #2}\\
	\end{tabular*}\vspace{-7pt}
}

\newcommand{\resumeSubItem}[1]{\resumeItem{#1}\vspace{-4pt}}

\renewcommand\labelitemi{$\vcenter{\hbox{\tiny$\bullet$}}$}
\renewcommand\labelitemii{$\vcenter{\hbox{\tiny$\bullet$}}$}

\newcommand{\resumeSubHeadingListStart}{\begin{itemize}[leftmargin=0.0in, label={}]}
		\newcommand{\resumeSubHeadingListEnd}{\end{itemize}}
\newcommand{\resumeItemListStart}{\begin{itemize}}
		\newcommand{\resumeItemListEnd}{\end{itemize}\vspace{-5pt}}


\newcommand{\Cpp}{C\texttt{++}}
\newcommand{\Csharp}{C\texttt{\#}}



% Define the skill command using tabularx to space out elements
\newcommand{\cvskill}[2]{%
	\noindent
	\begin{tabularx}{140pt}{@{}Xr@{}}
		\textbf{#1} & \skillstars{#2} \\
	\end{tabularx}%
}

% Define command for producing stars based on skill value
\newcommand{\skillstars}[1]{%
	\foreach \x in {1,...,5}{%
			\ifnum #1<\x
				{\color{base02}\faStar}% Empty stars from the skill level to 5
			\else
				{\color{base0A}\faStar}% Filled stars up to the skill
			\fi
		}%
}

%-------------------------------------------
